
\section{\label{sec:Conclusions-and-Future-Work}Conclusions and Future Work}

In this work we have suggested a novel self-stabilizing hypervisor
architecture which is capable of performing a wide range of stabilizing
actions starting from coarse-grained ones (like killing suspicious
VMs) to fine-grained ones (e.g., stopping guest applications whose
behavior seems to be suspicious). We have provided the conceptual
description of the architecture as well as the formal proof of its
self-stabilization. As a proof of concept we implement the self-stabilization
functionality in the KVM hypervisor \cite{kvm-site}. The design proposed
in this report adds the self-stabilization feature to the common hypervisor
architecture. We have also reviewed existing industrial products related
to the hypervisor level as a first step towards extending the current
work from a local node to a full distributed cloud setting. Finally,
we note that our implementation was tested with several initial denial
of service and rootkits attacks and exhibit low overhead in consistency
monitoring and guarded commands execution for recovering the system
to regain consistency. We plan to carry out more extensive experiments
using known test workloads like the rootkit repositories mentioned
in \cite{Wang:rootkits:2009} and to collect results.

Several future research directions we propose are provided below: 
\begin{itemize}
\item Executing guarded commands in a stabilization-preserving way. So far
our prototype has several defense policies (Denial of Service (DOS)
and rootkit defenses) hardcoded. We intend to support user-defined
specifications of defense policies. We intend also to build a compiler
that will transform our specifications into the behaviorally equivalent
set of actions while preserving stabilization.
\item Specifying device drivers formally. Our architecture is a part of
the Linux system and thus is not object-oriented. To bridge the gap
between the conceptions of our architecture (which is not object-oriented)
and UML (which describes object-oriented conceptions) we used stereotyping,
making specifications more informal. We intend to learn how to map
OS-related concepts into the OO domain following the literature (like
\cite{sertic-rus-rac:uml-realtime:2003}) and apply the approaches
to our specification thus making the latter more rigorous. 
\item Guests issuing DOS attacks. One of main challenges posed when detecting
DOS attacks is 'moving attackers', i.e. the party issuing DOS packets
(attacker) frequently changes the source IP. Source-based attacker
spotting is therefore complicated. Moving attacks in cloud infrastructures
can be easily mounted by placing attackers on migrating VMs (see \cite{key:article}
for further details on VM migration techniques). Making the Traffic
Monitor capable of recognizing various patterns (e.g., periodic bursts)
could improve the precision of DOS recognition. 
\item Comprehensive Security Model. The security model supported in our
architecture covers several common threats/failures plaguing cloud
infrastructures: VM state corruption, malicious inter-VM communication
and greedy resource allocation. We intend to study various security
models used in cloud infrastructures, create a comprehensive taxonomy
of VM threats and enhance our security model. \end{itemize}

