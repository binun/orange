
\section{Introduction}

Thc core entity used for facilitating virtualization in cloud computing
infrastructures is the \textit{hypervisor} (also referred to as the
\textit{VMM} - Virtual Machine Monitor) , as in \cite{Azab:2010:HES:1866307.1866313,Ben-Yehuda:nested-HV:2010,coker-sandboxing,Colp:HV-modularization:2011,DBLP:conf:raid:Stewin13,ddos-cpuid,defense-rootkit-attacks,GMU-CS-TR-Evasion-2011-8,hypervisor-security-future,liu-mao:distributed-HV:2013,McCune:2008:FEI:1357010.1352625,Pearce:2013:VIS:2431211.2431216}.
Being a basic part in the virtualization infrastructure the hypervisor
is the most attractive target for attackers. The (steadily rising)
complexity of hypervisors (\cite{Colp:HV-modularization:2011}) and
unlimited privileges of hypervisors over virtual machines%
\footnote{Abbreviated as \textit{VM}s in the article%
} (\cite{pfoh-vmmonitor-2013,Wailly:2012:VMS:2371536.2371564}) aggravate
the situation because a successful attack against the hypervisor almost
certainly brings the whole system down. In the meantime threats evolve
faster than hypervisor defense mechanisms; thus a significant part
of the attacks against the hypervisor succeed. This happens because
pro-active detection of malicious activities often requires the logic
of security modules to be guided by threat-specific behavioral patterns.
Detecting threats following the symptoms of (already compromised and
malfunctioning) system is a more generic task as in \cite{Wang:rootkits:2009}.

Trying to keep the situation under control, designers augment hypervisors
with additional security functions that are intended to fight new
threats. This functionality can set an additional security layer threats
attempt to compromise (for example, sandboxing of individual drivers
\cite{coker-sandboxing} or even running the whole hypervisor in an
additional VM , as in \cite{Ben-Yehuda:nested-HV:2010}). Another
way to fight malicious activities is sanitizing the internal structures
of (possibly already compromised) hypervisor, as in \cite{Wailly:2012:VMS:2371536.2371564}
thus mitigating the malicious actions. In any case, once a threat
captures the system, the latter faces severe danger, possibly crashing.

Here comes our novelty. Admitting that threats evolve faster than
the corresponding defense mechanisms, we shift the focus from preventing/mitigating
of malicious activities to building a system that is capable of recovering
gracefully after threat attacks and regaining stable behavior. We
refer to recovery as the existence of (at least) the possibility for
restarting from an initial (stable) state. Graceful recovery is a
recovery that tries to keep the system requirements during convergence
and converge fast to a stable state, possibly by rolling back (or
forward) restarting the system from the stable state that is nearest
in the execution history (which can be built by snapshotting and consistency
checks prior to reloading). In contrast to proactive defense, this
approach does not require the security logic to be guided by threat-specific
behavior and thus can be easily adapted to new threats.

We also leverage on the solid background in the area of self-stabilizing
systems. Following \cite{bib297:Dolev:selfstab,4553720,bib459:SS2004,Dolev:2009:SPC:1552309.1552312}
a system is self-stabilizing if every its execution always ends up
in a stable state after a finite number of steps no matter what state
it is initialized in. Stable (also called safe) states are distinguished
from unstable ones solely through the concrete application logic,
namely, any system execution that starts in a stable state exhibits
the desired application (also called task) behavior. 


\paragraph{Main Idea}

Materializing the above idea, we suggest a novel self-stabilizing
hypervisor architecture. Once in a time period, a special routine,
the \textit{stabilization manager} (SM) examines the hypervisor, checking
whether the latter is in a stable state. If needed, the system is
set into a safe (stable) state. The corresponding enforcement actions
range from simple, coarse-grained ones (like restarting suspicious
VMs or even the entire hypervisor) to fine-grained ones; for example,
stopping individual guest applications with suspicious behavior).
Upon success, the SM notifies the \textit{system watchdog} (explained
next) by sending the \textit{I\_Am\_Alive} message. Even if the SM
itself is corrupted (e.g., due to a successful attack) and thus does
not fulfill its duties, the watchdog ultimately reveals this (either
by the absence of \textit{I\_Am\_Alive} messages or following a system
integrity check) and restarts the system. The watchdog module is tamper-resilient
because it is write-protected (by virtue of residing in the hardware-protected
memory) and is triggered by hardware timer. The watchdog thus forms
the trusted computing base of our system. We provide a conceptual
description of the architecture and argue that it is self-stabilizing.

As a proof of concept we implement the SM as a separate Linux kernel
module collaborating with the KVM infrastructure \cite{kvm-site}.
We use KVM as a minimalistic hypervisor to illustrate the application
of our concept. We chose KVM because it is widely used, compact, simple
and open-sourced (being a part of the constantly maintained Linux
kernel). The last feature allows augmenting KVM with robustness functionality
in the coming stages.

Section \ref{sec:The-architecture} outlines the architecture of the
self-stabilizing hypervisor, arguing that our system is self-stabilizing.
Section \ref{sec:The-prototype} describes the prototype we are developing
as the proof of concept. Related work is sketched in Section \ref{sec:Related-Work}.
Finally, Section \ref{sec:Conclusions-and-Future-Work} presents the
conclusions of our work and outlines some future research directions. 
